% !TeX root = ../tfg.tex
% !TeX encoding = utf8

\chapter{Componentes de los algoritmos implementados}

En esta sección explicaremos con mayor profundidad la evolución diferencial, idea básica de los algoritmos que implementaremos más adelante, además explicaremos el funcionamiento de los algoritmos de agrupamiento diferencial y exploraremos dos métodos de búsqueda local que luego se utilizarán en el algoritmo SHADE-ILS

\section{Evolución diferencial}

Esta sección describe brevemente la Evolución Diferencial (DE). Similar a otros algoritmos evolutivos para la optimización numérica, una población de DE se representa como un conjunto de vectores de parámetros reales \( \mathbf{x}_i = (x_{1}, \ldots, x_{D}), i = 1, \ldots, N \), donde \( D \) es la dimensionalidad del problema objetivo y \( N \) es el tamaño de la población. 

Al comienzo de la búsqueda, los vectores individuales de la población se inicializan aleatoriamente. Luego, se repite un proceso de generación de vectores de prueba y selección hasta que se encuentra algún criterio de parada. En cada generación \( G \), se genera un vector mutado \( \mathbf{v}_{i,G} \) a partir de un miembro de la población existente \( \mathbf{x}_{i,G} \) aplicando alguna estrategia de mutación. A continuación se muestran ejemplos de estrategias de mutación:

\begin{itemize}
    \item \textbf{rand/1}
    \[
    \mathbf{v}_{i,G} = \mathbf{x}_{r1,G} + F \cdot (\mathbf{x}_{r2,G} - \mathbf{x}_{r3,G})
    \]

    \item \textbf{rand/2}
    \[
    \mathbf{v}_{i,G} = \mathbf{x}_{r1,G} + F \cdot (\mathbf{x}_{r2,G} - \mathbf{x}_{r3,G}) + F \cdot (\mathbf{x}_{r4,G} - \mathbf{x}_{r5,G})
    \]

    \item \textbf{best/1}
    \[
    \mathbf{v}_{i,G} = \mathbf{x}_{best,G} + F \cdot (\mathbf{x}_{r1,G} - \mathbf{x}_{r2,G})
    \]

    \item \textbf{current-to-best/1}
    \[
    \mathbf{v}_{i,G} = \mathbf{x}_{i,G} + F \cdot (\mathbf{x}_{best,G} - \mathbf{x}_{i,G}) + F \cdot (\mathbf{x}_{r1,G} - \mathbf{x}_{r2,G})
    \]
\end{itemize}

Los índices \( r1, \ldots, r5 \) se seleccionan aleatoriamente de \([1, N]\) de manera que difieran entre sí, así como \( i \). \( \mathbf{x}_{best,G} \) es el mejor individuo en la población en la generación \( G \). El parámetro \( F \in [0, 1] \) controla la potencia del operador de mutación diferencial, a mayor F, mayor será la diferencia entre el vector original y el mutado.

Después de generar el vector mutado \( \mathbf{v}_{i,G} \), se cruza con el padre \( \mathbf{x}_{i,G} \) para generar el vector de prueba \( \mathbf{u}_{i,G} \). El cruce binomial, el operador de cruce más utilizado en DE, se implementa de la siguiente manera:

\[
u_{j,i,G} =
\begin{cases}
v_{j,i,G} & \text{si } \text{rand}[0, 1) \leq CR \text{ o } j = j_{\text{rand}} \\
x_{j,i,G} & \text{de lo contrario}
\end{cases}
\]

\text{rand}[0, 1) denota un número aleatorio seleccionado uniformemente de \([0, 1)\), y \( j_{\text{rand}} \) es un índice de variable de decisión que se selecciona aleatoriamente y de manera uniforme de \([1, D]\). \( CR \in [0, 1] \) es la tasa de cruce.

Después de que se han generado todos los vectores de prueba \( \mathbf{u}_{i,G} \), un proceso de selección determina los supervivientes para la siguiente generación. El operador de selección en DE estándar compara cada individuo \( \mathbf{x}_{i,G} \) contra su vector de prueba correspondiente \( \mathbf{u}_{i,G} \), manteniendo el mejor vector en la población.

\[
\mathbf{x}_{i,G+1} = 
\begin{cases} 
\mathbf{u}_{i,G} & \text{si } f(\mathbf{u}_{i,G}) \leq f(\mathbf{x}_{i,G}) \\ 
\mathbf{x}_{i,G} & \text{de lo contrario} 
\end{cases} 
\]

\section{Búsqueda local}

\section{Algoritmos de descomposición}

\endinput
%--------------------------------------------------------------------
% FIN DEL CAPÍTULO. 
%--------------------------------------------------------------------
