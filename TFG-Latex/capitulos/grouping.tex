% !TeX root = ../tfg.tex
% !TeX encoding = utf8

\chapter{Agrupamiento de variables}

Inspirados por la metodología del análisis clúster, estudiaremos los fundamentos teóricos del agrupamiento diferencial, una técnica de agrupamiento de variables que permite descomponer un problema en subproblemas menores. Esta metodología pretende dividir un conjunto de variables acorde a la interdependencia que se establece entre ellas cuando se trata de optimizar una función objetivo.

Al igual que en el análisis clúster, que agrupa conjuntos de datos en clústers de forma que en cada clúster los datos sean lo más parecidos posibles y distintos del resto de clúster, el objetivo de esta técnica es separar las variables en conjuntos, de forma que cada conjunto sea independiente del resto y dentro de cada conjunto ninguna variable sea independiente. La principal diferencia radica en que en el análisis clúster agrupamos datos acorde al valor de las variables y en el agrupamiento de variables lo que agrupamos son las variables acorde a la dependencia que existe entre ellas. 

A continuación, se exponen las definiciones y teoremas necesarios para entender el agrupamiento diferencial desde un punto de vista teórico. En la segunda parte de este TFG, se implementarán distintas variantes de esta técnica para probar su efectividad a la hora de hibridarlas con algoritmos que permitan optimizar una función objetivo.

\section{Definiciones}


\section{Teoremas}

\section{Problemas descomponibles mediante agrupamiento de variables}


\endinput
%--------------------------------------------------------------------
% FIN DEL CAPÍTULO. 
%--------------------------------------------------------------------
