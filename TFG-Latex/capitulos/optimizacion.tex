% !TeX root = ../tfg.tex
% !TeX encoding = utf8

\chapter{Preliminares}

\textbf{Derivada Direccional:} Sea \( f : \mathbb{R}^n \to \overline{\mathbb{R}} \) una función diferenciable, y \( \mathbf{u} = (u_1, \ldots, u_n) \) un vector de \( U_X \). La derivada direccional de \( f \) en la dirección \( \mathbf{u} \), denotada \( D_{\mathbf{u}} f(\mathbf{x}) \), se define como
\[
D_{\mathbf{u}} f(\mathbf{x}) = \sum_{i=1}^{n} \frac{\partial f(\mathbf{x})}{\partial x_i} u_i.
\]

\textbf{Integral de Línea:} Sea \( L \) una curva con puntos extremos \( A \) y \( B \) en el espacio de decisión \( \mathbb{R}^n \) y la longitud del arco de \( L \) sea \( l \). Sea \( C \) cualquier punto en \( L \) y la coordenada de \( C (\mathbf{x}) \) puede ser determinada de manera única por la longitud del arco \( AC (s) \): \( \mathbf{x} = \mathbf{x}(s) \), \( s \in [0, l] \). La integral de una función \( g : \mathbb{R}^n \to \overline{\mathbb{R}} \) a lo largo de la curva \( L \) se da por
\begin{equation}
\int_L g(\mathbf{x}) ds = \int_0^l g(\mathbf{x}(s)) ds.
\label{EQ0}
\end{equation}

\chapter{Optimización numérica}

\section{Definiciones}

\section{Teoremas}

\section{Dificultades(Alta dimensionalidad)}

\section{¿Evolución Diferencial?}

\endinput
%--------------------------------------------------------------------
% FIN DEL CAPÍTULO. 
%--------------------------------------------------------------------
