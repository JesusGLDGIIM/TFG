% !TEX program = pdflatex
% !TEX encoding = UTF-8 Unicode

% Plantilla, basada en la clase `scrbook` del paquete KOMA-script,  para la elaboración de un TFG siguiendo las directrices del la comisión del Grado en Matemáticas de la Universidad de Granada.

% Francisco Torralbo Torralbo

\documentclass[print, color]{ugrTFG}

\usepackage[utf8]{inputenc}
\usepackage[numbers]{natbib} % Añadir opción numbers
\usepackage{url}
\usepackage{amsmath}
\usepackage{algorithm}
\usepackage{algorithmic}
\usepackage{caption}

% VERSIÓN ELECTRÓNICA PARA TABLETA
% Cambiando la opción "print" por "tablet" generaremos un pdf adaptado para leerlo en tabletas de 9 pulgadas.

% -------------------------------------------------------------------
% INFORMACIÓN DEL TFG Y EL AUTOR
% -------------------------------------------------------------------

\newcommand{\miTitulo}{Combinando distintas técnicas para el diseño de una metaheurística para problemas de optimización de alta dimensionalidad \xspace}
\newcommand{\miNombre}{Jesús García León\xspace}
\newcommand{\miGrado}{Doble Grado en Ingeniería Informática y Matemáticas}
\newcommand{\miFacultad}{Escuela Técnica Superior de Ingeniería Informática y de Telecomunicación y Facultad de Ciencias}
\newcommand{\miUniversidad}{Universidad de Granada}

% Añadir tantos tutores como sea necesario separando cada uno de ellos mediante el comando `\medskip` y una línea en blanco
\newcommand{\miTutor}{Daniel Molina Cabrera \\ \emph{Departamento de Ciencias de la Computación e Inteligencia Artificial \vspace{10px}} 

  % Añadir tantos tutores como sea necesario. 

  %\medskip
  %Nombre del tutor 2 \\ \emph{Departamento del tutor 2}
}
\newcommand{\miCurso}{2023-2024\xspace}

\hypersetup{
	pdftitle={\miTitulo},
	pdfauthor={\textcopyright\ \miNombre, \miFacultad, \miUniversidad}
}

\begin{document}

\maketitle

% -------------------------------------------------------------------
% FRONTMATTER
% -------------------------------------------------------------------
\frontmatter % Desactiva la numeración de capítulos y usa numeración romana para las páginas

\input{preliminares/declaracion-originalidad}   
%\input{preliminares/dedicatoria}                % Opcional
% !TeX root = ../tfg.tex
% !TeX encoding = utf8

%*******************************************************
% Agradecimientos
%*******************************************************

\chapter{Agradecimientos}

Quiero agradecer en primer lugar a mis padres, ya que sin su esfuerzo y apoyo nada de este trabajo habría sido posible. A mis compañeros de clase por permanecer a mi lado durante todo el camino, motivarme a aprender más cada día para no quedarme atrás, por enseñarme y por dejarse enseñar. A mis amigos por todo el apoyo incondicional que me han dado en los buenos y malos momentos, un apoyo no solo académico, sino moral, que ha sido necesario para poder llegar hasta aquí. Por último quiero agradecer a mi tutor, por aconsejarme y guiarme a lo largo de este trabajo.

\cleardoublepage
\endinput
            % Opcional
\input{preliminares/tablacontenidos}            
% !TeX root = ../tfg.tex
% !TeX encoding = utf8
%
%*******************************************************
% Resumen
%*******************************************************

\chapter{Resumen}

\indent El aumento de la capacidad de cálculo de los ordenadores ha posibilitado resolver problemas cada vez más complejos, que dependen de un número creciente de variables. La utilización de algoritmos clásicos en este contexto no es posible o resulta extremadamente costosa debido al incremento exponencial en la dificultad que trae consigo la alta dimensionalidad. Esto hace conveniente diseñar algoritmos capaces de encontrar soluciones para problemas de alta dimensionalidad. Este problema se conoce como \textit{la maldición de la alta dimensionalidad}. Este fenómeno es el aumento exponecial del espacio de búsqueda que se produce al aumentar la dimensionalidad de un problema, lo que dificulta gravemente la búsqueda de soluciones.

En problemas de alta dimensionalidad, muchas técnicas que funcionan eficientemente en espacios de baja dimensión se vuelven ineficaces, ya que los datos ocupan solo una pequeña fracción del espacio total. Esto afecta algoritmos de búsqueda, clasificación y optimización, ya que el aumento del espacio de búsqueda, trae consigo un aumento del tiempo necesario, que hace inviable evaluar todas las posibles opciones.

En este trabajo proponemos el estudio e implementación de varias técnicas de \textbf{agrupamiento de variables}, que busca separar las variables de un problema en subgrupos independientes, permitiendo optimizar cada subgrupo por separado. Esto reduce significativamente la dimensionalidad efectiva del problema, permitiendo una optimización más eficiente. Al dividir el conjunto de variables en subgrupos independientes, aprovechamos el paralelismo y reducimos la complejidad computacional, ya que cada subgrupo puede optimizarse sin afectar a los demás. Esta técnica es especialmente útil en problemas donde ciertas variables están altamente correlacionadas entre sí, pero tienen poca o ninguna relación con otras variables del conjunto.

Estudiaremos esta técnica aplicándola a algoritmos diseñados para resolver problemas de alta dimensionalidad y a algoritmos que funcionan mejor en baja dimensión. Así, podremos comparar si es más eficiente diseñar algoritmos que mejoren en alta dimensionalidad o dividir el problema en subproblemas menores. Para ello, introduciremos los conceptos matemáticos necesarios para entender cómo resolver problemas de minimización, así como la teoría detrás de los algoritmos de agrupamiento de variables. Además, analizaremos qué tests estadísticas podemos emplear para obtener una comparación objetiva y efectiva, explicando su selección.

Presentaremos también algoritmos metaheurísticos básicos que más adelante combinaremos para crear nuestra propuesta. Entre ellos, se incluyen \textit{SHADE} y \textit{SHADE-ILS} como algoritmos de optimización, y \textit{ERDG} como algoritmo de agrupamiento de variables.

Adicionalmente, desarrollaremos una biblioteca en el lenguaje de programación \textit{Julia} que implemente los algoritmos utilizados. Esta biblioteca no solo tiene como objetivo evaluar la efectividad de la técnica de agrupamiento de variables, sino también contribuir a la comunidad de \textit{Julia}, proporcionando una herramienta que otros desarrolladores podrán utilizar y mejorar en el futuro. Para comparar los resultados, utilizaremos el benchmark \textit{CEC2013 LSGO}, diseñado específicamente para problemas de optimización global de alta dimensionalidad.

\textbf{Palabras clave:} capacidad de cálculo, optimización, redes neuronales, alta dimensionalidad, maldición de la dimensionalidad, agrupamiento de variables, complejidad computacional, metaheurísticas, Julia.

\endinput


% !TeX root = ../tfg.tex
% !TeX encoding = utf8
%
%*******************************************************
% Summary
%*******************************************************

\selectlanguage{english}
\chapter{Abstract}

\indent The increase in computing power has driven the need to solve increasingly complex problems that depend on a larger number of variables, making it necessary to design algorithms capable of finding solutions to high-dimensional problems. For example, the optimization of neural networks often involves optimizing hundreds or even millions of variables, which is either impossible or extremely costly using classical algorithms due to the exponential increase in difficulty associated with higher dimensionality. This problem is primarily due to \textit{the curse of dimensionality}. This term refers to the difficulties that arise as the number of dimensions (or variables) in a space increases, negatively impacting the performance and effectiveness of optimization and analysis algorithms. As dimensionality grows, distances between points become less informative, the space's volume increases exponentially, and data becomes increasingly sparse, making analysis more challenging.

In high-dimensional problems, many techniques that work well in low-dimensional spaces become ineffective because the data occupies only a small fraction of the entire space. This impacts algorithms for search, classification, and optimization, which require more time and resources to process and evaluate all possible combinations of variables.

In this work, we propose studying a \textbf{variable grouping} technique that aims to separate the variables of a problem into independent subgroups, allowing each subgroup to be optimized separately. This significantly reduces the effective dimensionality of the problem, enabling more efficient optimization. By dividing the set of variables into independent subgroups, we can leverage parallelism and reduce computational complexity, as each subgroup can be optimized without directly affecting the others. This technique is particularly useful for problems where certain variables are highly correlated with each other but have little or no relation to other variables in the set.

We will study the use of this technique in algorithms specifically designed to solve high-dimensional problems, as well as in algorithms that work better in low dimensions. This will allow us to compare whether it is more efficient to focus on designing algorithms that perform increasingly well in high-dimensional spaces or to break the problem down into smaller subproblems.

To achieve this, we will introduce the necessary mathematical concepts for understanding how to solve minimization problems, as well as the theory behind variable grouping algorithms. We will also analyze which statistical tests we can use to obtain an objective and effective comparison and explain why they were chosen.

We will also present basic metaheuristic algorithms, which we will later combine to create our proposal. These include \textit{SHADE} and \textit{SHADEils} as optimization algorithms and \textit{ERDG} as a variable grouping algorithm.

Additionally, we will develop a library in the \textit{Julia} programming language that implements the algorithms used. This library aims not only to demonstrate whether the variable grouping technique is effective but also to contribute to the \textit{Julia} community by providing a tool that other developers can use and improve upon in the future.

To compare the results, we will use the \textit{CEC2013 LSGO} benchmark, specifically designed for large scale global optimization problems. For results analysis, we will use the \textit{Tacolab} web platform, which allows for straightforward comparisons between algorithms. Based on the data obtained, we will objectively determine in which cases this technique can be effective and in which it cannot.

Keywords: computing power, optimization, neural networks, high dimensionality, curse of dimensionality, variable grouping, parallelism, computational complexity, metaheuristic algorithms, Julia, benchmark, Tacolab.

% Al finalizar el resumen en inglés, volvemos a seleccionar el idioma español para el documento
\selectlanguage{spanish} 
\endinput
      
% !TeX root = ../tfg.tex
% !TeX encoding = utf8
%
%*******************************************************
% Presupuesto
%*******************************************************

\chapter{Presupesto}

Aquí irá el presupuesto estimado del proyecto, basándose en el sueldo de un programador que deberíamos contratar para realizar el proyecto en función del número de horas y el coste de los servidores necesarios para realizar los cálculos de los algoritmos utlizados.

File: \texttt{preliminares/presupuesto.tex}

\endinput              
% !TeX root = ../tfg.tex
% !TeX encoding = utf8
%
%*******************************************************
% Introducción
%*******************************************************

% \manualmark
% \markboth{\textsc{Introducción}}{\textsc{Introducción}} 

\chapter{Introducción}

\indent Cuando enfrentamos un problema de optimización que depende de pocos factores, es sencillo idear estrategias para resolverlo y encontrar una solución. Sin embargo, a medida que crece el número de variables, la complejidad del problema aumenta, y se hace inviable resolverlo mediante métodos tradicionales. Para abordar esta problemática, se han diseñado algoritmos específicos para manejar la alta dimensionalidad utilizando metaheurísticas. Estas buscan optimizar la exploración del espacio de búsqueda, logrando un equilibrio entre exploración y explotación, lo que permite realizar búsquedas efectivas en un espacio donde la porción representada por la población de búsqueda es ínfima comparada con el espacio total. No obstante, estos algoritmos presentan limitaciones, y cuando el número de variables es suficientemente alto, debemos explorar otras soluciones.

En este trabajo, proponemos utilizar la técnica del \textbf{agrupamiento diferencial de variables}, que permite agrupar las variables en conjuntos, de forma que cada variable interactúa principalmente con las de su propio grupo y no (o mínimamente) con las de otros grupos. Así, podemos reducir la dimensión efectiva del problema, convirtiendo un problema de optimización de gran escala en problemas de optimización de menor tamaño. Nuestro objetivo es hibridar esta técnica con algoritmos diseñados para la alta dimensionalidad y con algoritmos que no están específicamente adaptados para ello. Compararemos el rendimiento aplicando y no aplicando esta técnica en ambos tipos de algoritmos.

Para probar la efectividad de esta técnica, se han planteado los siguientes objetivos:

\section*{Objetivos}

\subsection*{Parte matemática}
\begin{itemize}
    \item \textbf{Análisis teórico}: Repaso de las técnicas matemáticas tradicionales en la literatura para resolver problemas de optimización sin restricciones.
    \item \textbf{Estudio de teoremas}: Exposición y análisis de los teoremas en los que se basa el agrupamiento diferencial.
    \item \textbf{Fundamentos estadísticos}: Explicación de los tests estadísticos utilizados para la comparación de algoritmos.
\end{itemize}

\subsection*{Parte informática}
\begin{itemize}
    \item \textbf{Desarrollo de una biblioteca de algoritmos}: Implementación de algoritmos para optimización continua sin restricciones en el lenguaje de programación Julia.
    \item \textbf{Selección de algoritmos}: Descripción de los algoritmos que se incluirán en la biblioteca y que se emplearán en las comparaciones.
    \item \textbf{Evaluación experimental}: Análisis de los algoritmos en el benchmark \textit{cec2013lsgo}, presentación de resultados y conclusiones obtenidas.
\end{itemize}

\section*{Estructura de la memoria}

El trabajo se divide en tres partes claramente diferenciadas y cada parte está dividida en capítulos. Ademas de las dos partes que constituyen el contenido de este TFG, tenemos esta primera sección introductoria, donde se pone en contexto el trabajo, se definen los objetivos y se hace un repaso de la bibliografía necesaria. 
Además, se hace una estimación del presupuesto y del tiempo dedicado a cada tarea.

La parte Matemática está dividida en 3 capítulos, un primer capítulo donde se realizan definiciones importantes sobre optimización de funciones, se exponen algunos teoremas necesarios y se explican los métodos de búsqueda lineal y el algoritmo L-BFGS-B, que será un componente importante de nuestro algoritmo final. El segundo capítulo de esta sección se define el agrupamiento diferencial y se citan y demuestran los teoremas más importantes que justifican el diseño y utilidad de los algoritmos de agrupamiento automático de variables. Además se proporcionan algunas definiciones previas necesarias sobre separabilidad de funciones. Por último en la tercera sección de esta parte, se explican los fundamentos de los tests estadísticos y se definen los tests que suelen ser utilizados para comparar algoritmos. 

En la parte informática, tenemos un capitulo inicial donde se definen los componentes de los algoritmos que formarán nuestra propuesta, una segunda parte donde se explican los algoritmos que luego hibridaremos con el agrupamiento de variables. En tercer lugar se explicará nuestra propuesta: combinar ERDG con SHADE y SHADEils. Por último, unas secciones finales donde se exponen los resultados obtenidos y las conclusiones que hemos obtenido de los resultados.

\section*{Contexto Bibliográfico}

En esta sección, haremos un repaso bibliográfico sobre la optimización global a gran escala (LSGO, por sus siglas en inglés, \textit{Large Scale Global Optimization}). Se proporcionará un contexto general de las distintas técnicas metaheurísticas y enfoques que existen para abordar este tipo de problemas, y se situarán nuestros algoritmos en este contexto, explicando en mayor profundidad aquellas técnicas que están más relacionadas con los algoritmos \textit{SHADE}, \textit{SHADEILS}, \textit{ERDG} y \textit{DG2}.

\subsection*{Introducción a la Optimización Global a Gran Escala}

La \textit{maldición de la dimensionalidad} ha sido un problema fundamental en la optimización global. Este problema se acentúa en la optimización global a gran escala, donde el número de variables y restricciones crece exponencialmente. El objetivo principal de las técnicas LSGO es mejorar la escalabilidad de los algoritmos de optimización a medida que el número de variables, \( n \), y su dimensionalidad crece.

\subsection*{Enfoques Principales en LSGO}

A continuación, se presentan los enfoques principales en LSGO, los cuales se han desarrollado para afrontar la complejidad de los problemas de optimización de gran escala:

\begin{enumerate}
    \item \textbf{Descomposición del problema}: Dividir el problema en subproblemas manejables.
    \item \textbf{Hibridación y búsqueda local memética}: Combina algoritmos evolutivos con técnicas de búsqueda local.
    \item \textbf{Operadores de muestreo y variación}: Técnicas de muestreo para explorar el espacio de búsqueda.
    \item \textbf{Modelado por aproximación y uso de modelos sustitutos}: Se utilizan modelos simplificados para reducir el coste computacional.
    \item \textbf{Métodos de inicialización}: Métodos para asegurar una cobertura uniforme del espacio de búsqueda.
    \item \textbf{Paralelización}: Uso de múltiples instancias de algoritmos para acelerar la búsqueda.
\end{enumerate}



\subsection*{Operadores de Muestreo y Variación}

Los operadores de muestreo y variación buscan mantener la diversidad en la población y mejorar la eficacia de los algoritmos en la exploración de grandes espacios de búsqueda. Dos enfoques comunes son:

\subsubsection*{Evolución Diferencial (DE)}

La Evolución Diferencial (DE) es un algoritmo popular en la optimización global debido a su simplicidad y efectividad. Variantes como \textit{SHADE} y \textit{SHADEILS} han surgido como adaptaciones de DE para problemas de gran escala:
\begin{itemize}
    \item \textbf{SHADE}: Una variante de DE que ajusta adaptativamente el tamaño de la población y los parámetros de mutación para mantener la diversidad en poblaciones grandes.
    \item \textbf{SHADEILS}: Extiende SHADE mediante la integración de estrategias de búsqueda local, mejorando la precisión en problemas de alta dimensionalidad.
\end{itemize}

\subsubsection*{Particle Swarm Optimization (PSO)}

PSO es un método basado en el comportamiento social de partículas. Aunque efectivo en problemas de baja dimensionalidad, enfrenta retos en alta dimensionalidad, para lo cual se han introducido estrategias de \textit{mantenimiento de diversidad} y \textit{re-inicialización}.

\subsection*{Modelado por Aproximación y Modelos Sustitutos}

Dado que resolver directamente un problema de gran escala puede ser costoso, el modelado por aproximación crea un modelo simplificado de la función objetivo. Este enfoque es particularmente útil cuando se dispone de una función de alto coste computacional.

\subsection*{Métodos Explícitos para el Aprendizaje de Interacción de Variables}

Los métodos explícitos para el aprendizaje de interacción buscan explotar las relaciones entre variables. Esto es fundamental en problemas donde la estructura del problema puede ser aprovechada.

\subsubsection*{ERDG y DG2}

\begin{itemize}
    \item \textbf{ERDG (Recursive Differential Grouping)}: ERDG se basa en la identificación de interacciones mediante particionamiento iterativo de variables. Esta técnica permite detectar grupos de variables interdependientes, lo cual facilita la resolución de subproblemas de forma eficiente.
    \item \textbf{DG2}: Es una técnica de descomposición que aprovecha las diferencias finitas para identificar interacciones. A diferencia de ERDG, DG2 ofrece una mayor precisión en la identificación de interacciones con una complejidad computacional más baja.
\end{itemize}

ERDG y DG2 son cruciales en el contexto de LSGO, ya que permiten dividir el problema en componentes independientes, lo cual facilita el proceso de optimización en problemas de gran escala.

\subsection*{Ventajas y Desventajas de las Técnicas}

Según el teorema \textit{No Free Lunch}, no existe un algoritmo universal que sea óptimo para todos los problemas. Por lo tanto, los enfoques híbridos que combinan varias estrategias tienden a ofrecer mejores resultados.

\subsubsection*{Hibridación en Algoritmos Evolutivos}

La hibridación en algoritmos evolutivos permite:
\begin{enumerate}
    \item Mejorar la velocidad de convergencia.
    \item Incrementar la calidad de la solución.
    \item Incorporar el algoritmo en sistemas más grandes.
\end{enumerate}

%
De acuerdo con la comisión de grado, el TFG debe incluir una introducción en la que se describan claramente los objetivos previstos inicialmente en la propuesta de TFG, indicando si han sido o no alcanzados, los antecedentes importantes para el desarrollo, los resultados obtenidos, en su caso y las principales fuentes consultadas.
%


Ver archivo \texttt{preliminares/introduccion.tex}

\endinput
               

% -------------------------------------------------------------------
% MAINMATTER
% -------------------------------------------------------------------
\mainmatter % activa la numeración de capítulos, resetea la numeración de las páginas y usa números arábigos

\part{Parte Matemática} % Dividir un TFG en partes OPCIONAL

% !TeX root = ../tfg.tex
% !TeX encoding = utf8

\chapter{Problemas de Optimización}

\section{Definiciones}

\section{Algoritmos}

\section{Dificultades}

\endinput
%--------------------------------------------------------------------
% FIN DEL CAPÍTULO. 
%--------------------------------------------------------------------

% !TeX root = ../tfg.tex
% !TeX encoding = utf8

\chapter{Grouping}

\section{Definiciones}

\section{Teoremas}

\section{Problemas descomponibles mediante agrupamiento de variables}


\endinput
%--------------------------------------------------------------------
% FIN DEL CAPÍTULO. 
%--------------------------------------------------------------------

% !TeX root = ../tfg.tex
% !TeX encoding = utf8

\chapter{Algoritmos de optimización en alta dimensión}

\section{Definición}

\section{State of art}


\endinput
%--------------------------------------------------------------------
% FIN DEL CAPÍTULO. 
%--------------------------------------------------------------------

% !TeX root = ../tfg.tex
% !TeX encoding = utf8

\chapter{Tests estadísticos}

\section{Definición}

\section{Tests}
\subsection{Test 1}
\subsection{Test 2}
\subsection{Test 3}

\section{Relevancia en el contexto de este TFG}

\endinput
%--------------------------------------------------------------------
% FIN DEL CAPÍTULO. 
%--------------------------------------------------------------------


% Información relevante para la elaboración del trabajo.
%\input{capitulos/documentacion}
%\input{capitulos/recomendaciones}

% Añadir tantos capítulos como sea necesario

\cleardoublepage\part{Parte Informática}

% !TeX root = ../tfg.tex
% !TeX encoding = utf8

\chapter{Metahehurísticas}

\section{Definición}
Explicar las metaheuristicas en general, diferencias con las heuristicas y su clasificación. Nos centraremos en los algoritmos evolutivos, en los de descomposición y de búsqueda local

\section{Algoritmos evolutivos}

\subsection{Evolución diferencial}

\section{Algoritmos de descomposición}

\section{Búsqueda local}



\endinput
%--------------------------------------------------------------------
% FIN DEL CAPÍTULO. 
%--------------------------------------------------------------------

% !TeX root = ../tfg.tex
% !TeX encoding = utf8

\chapter{Algoritmos de comparación}

Se explican los algoritmos que combinaremos para crear nuestra propuesta y que se utilizarán también para comparar como mejora el algoritmo final con los algoritmos básicos. Se incluirá el pseudocódigo y la explicación de las partes esenciales que componen cada algortimo.

\section{SHADE}

\section{SHADE-ILS}

\section{DG2}

\section{RDG2}

\endinput
%--------------------------------------------------------------------
% FIN DEL CAPÍTULO. 
%--------------------------------------------------------------------

% !TeX root = ../tfg.tex
% !TeX encoding = utf8

\chapter{Propuesta}

\section{DG2-SHADE-ILS}

\section{ERDG-SHADE-ILS}

\endinput
%--------------------------------------------------------------------
% FIN DEL CAPÍTULO. 
%--------------------------------------------------------------------

% !TeX root = ../tfg.tex
% !TeX encoding = utf8

\chapter{Resultados}

\section{Benchmark}
\cite{cec_2013_lsgo}

\section{Resultados obtenidos}


\endinput
%--------------------------------------------------------------------
% FIN DEL CAPÍTULO. 
%--------------------------------------------------------------------

% !TeX root = ../tfg.tex
% !TeX encoding = utf8

\chapter{Conclusiones y futuras lineas de investigación}
En este último capítulo, se resumen las conclusiones extraídas del análisis de los datos obtenidos y se proponen posibles líneas de investigación basadas en las conclusiones de por qué al introducir agrupamiento de variables puede no estar funcionando.

Hemos comprobado que aplicar un agrupamiento previo de las variables, es ciertamente efectivo cuando se utiliza en el marco de la coevolución cooperativa, confirmando lo que afirmaban los autores en su publicación original. También hemos reafirmado la necesidad que tenían los autores por encontrar un algoritmo de agrupamiento que fuese mas efectivo que el agrupamiento diferencial, como el ERDG, ya que DG2 consumía demasiadas evaluaciones, lo que lo hacía inviable en problemas de muy alta dimensionalidad o en problemas que requiriesen de una respuesta más rápida.

Sin embargo, cuando tratamos de aplicar el agrupamiento de variables a algoritmos diseñados especialemente para la alta dimensionalidad como SHADE-ILS, los resultados fueron decepcionantes, ya que en la mayoría de los casos, los resultados obtenidos fueron peores. A pesar de ello, obtuvimos información que puede ser de utilidad para crear una propuesta mejor en el futuro.

\begin{itemize}
	\item Como en la coevolución cooperativa si que se obtienen mejores resultados, sería una buena idea integrar SHADE-ILS en este marco, para comprobar si se obtienen mejores resultados.

	\item Comparando la asignación estática de evaluaciones con la evaluación ponderada, comprobamos que en la mayoría de los casos la ponderada era más eficaz, sin embargo, en ciertos casos la estática era más efectiva, lo que nos incita a probar una asignación de recursos diferente, que no se base únicamente en el tamaño de cada grupo si no en la contribución que tiene cada grupo en la mejora del resultado final, asignando más evaluaciones a aquellos que reporten mejores resultados durante el proceso de evolución. Por ejemplo se podría incluir en el marco de CCFR, lo que permitiría usar coevolución cooperativa con una asignación eficiente de recursos, solucionando los dos posibles problemas que hemos encontrado a la hora de realizar este trabajo.
	
	\item En problemas no separables, ERDG-SHADE si que presenta mejores resultados que SHADE cuando el número de evaluaciones es bajo, por tanto para problemas no separables que requieran de una respuesta rápida, la aplicación de esta técnica si que ha obtenido resultados favorables. También podría probarse introducir SHADE en CCFR para este tipo de problemas.
\end{itemize}

Por último, a pesar de que los resultados no fuesen favorables, hemos creado una biblioteca de código abierto en el lenguaje de programación Julia, acercando el código sobre agrupamiento de variables, SHADE y SHADE-ILS a esta comunidad, lo que contribuirá a aumentar la riqueza de este lenguaje y permitirá que se puedan realizar futuros experimentos en este campo de forma más accesible, puesto que ahora se dispone de una base sobre la que experimentar y no es necesario partir de los algoritmos originales implementados en MATLAB. Esta biblioteca está disponible en github \cite{biblioteca}

\endinput
%--------------------------------------------------------------------
% FIN DEL CAPÍTULO. 
%--------------------------------------------------------------------


%\input{capitulos/capitulo-ejemplo}

% -------------------------------------------------------------------
% APPENDIX: Opcional
% -------------------------------------------------------------------

\appendix % Reinicia la numeración de los capítulos y usa letras para numerarlos
\pdfbookmark[-1]{Apéndices}{appendix} % Alternativamente podemos agrupar los apéndices con un nuevo \part{Apéndices}

\input{apendices/apendice-ejemplo}
% Añadir tantos apéndices como sea necesario 

% -------------------------------------------------------------------
% GLOSARIO: Opcional
% -------------------------------------------------------------------

\input{glosario} 

% -------------------------------------------------------------------
% BACKMATTER
% -------------------------------------------------------------------

\backmatter % Desactiva la numeración de los capítulos
\pdfbookmark[-1]{Referencias}{BM-Referencias}

% BIBLIOGRAFÍA
%-------------------------------------------------------------------

\bibliographystyle{alpha-es} 
\begin{small} 
  \bibliography{library.bib}
\end{small}


\end{document}
